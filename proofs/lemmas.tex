\documentclass{article}

\usepackage{amsthm}
\newtheorem*{duplication}{Minimally Duplicated}
\newtheorem*{overhead}{Overhead Bar}
\newtheorem*{trees}{Unnecessary Trees}
\newtheorem{mydef}{Definition}

\begin{document}
%\title{Unorganized Lemmas}
%\author{Michael James \and Ben Hescott}
%\date{\today}
%\maketitle

%\begin{abstract}
%Here is where I'll introduce some stuff.
%\end{abstract}

\section{Proofs}
Any graph, \emph{G}, can be rewritten such that it has levels.  This is like a tree however there may be cycles.  More specifically, each vertex (or node) has parents, children, and siblings.

\theoremstyle{definition}
\begin{mydef}
A \textbf{minimum planar-expansion} is the planar-expansion of a graph, G, using the fewest number of duplications possible.
\end{mydef}

\begin{trees}
If some planar-expansion results in two independant trees with a common root vertex, then one of the trees is unnecessary for the planar expansion.\\
\textbf{Proof: }Suppose that the planar-expansion tree begins with the vertex $V_a$ and has children $V_b_1\ and\ V_b_2$.
\end{trees}

\begin{duplication}
Given some non-planar graph, G, if some vertex, $V_i$, must be duplicated to create a minimum planar-expansion, then all vertices must be duplicated.\\
\textbf{Proof: }For the sake of contradiction, assume not and that vertex $V_a$ is not duplicated to create a planar-expansion. $V_i$, however, was duplicated s.t. $\exists A = \left\{V_j,...,V_n\right\}$
\end{duplication}
\end{document}
