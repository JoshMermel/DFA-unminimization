\documentclass{article}

\usepackage{amsthm}
\newtheorem*{duplication}{Minimally Duplicated}
\newtheorem*{overhead}{Overhead Bar}
\newtheorem*{trees}{Unnecessary Trees}
\newtheorem*{mincolor}{0-Color Repetition}
\newtheorem*{symmetry}{Planar-expansion Symmetry}
%It looks like planar expansions have some element of symmetry to them.  This
%could be because of the duplication.  Thus this would only be be pretty
%corollary.
\newtheorem{mydef}{Definition}

\begin{document}
%\title{Unorganized Lemmas}
%\author{Michael James \and Ben Hescott}
%\date{\today}
%\maketitle

%\begin{abstract}
%Here is where I'll introduce some stuff.
%\end{abstract}

\section{Proofs}
Any graph, \emph{G}, can be rewritten such that it has levels.  This is like a 
tree however there may be cycles.  More specifically, each vertex (or node) has
parents, children, and siblings.\\

\theoremstyle{definition}
\begin{mydef}
Given some graph, \emph{G}, give every vertex its own color. Each color, $C_i$, 
is connected to colors $C_{j},...,C_{n}$. Now make copies of
each color such that the graph is planar and each $C_i$ has a relationship with 
at least one color for each $C_{j},...,C_{n}$.  The resulting graph is a 
\textbf{planar-expansion}.\\
\end{mydef}

\theoremstyle{definition}
\begin{mydef}
A \textbf{minimum planar-expansion} is the planar-expansion of a graph, G, using
the fewest number of edges and vertices as possible while maintaining the 
properties of a planar-expansion.\\
\end{mydef}


\begin{trees}
If some planar-expansion results in two independent, minimally expanded trees 
where the root vertices are the same color, then it is not a minimal 
planar-expansion.
\end{trees}
\paragraph{Proof: }
Assume not, and that there is some vertex or edge in one of the two trees, 
$V_b$ or $V_c$ that is required for the common parent, $V_a$ to be satisfied.
Because $V_b$ and $V_c$ share the same color, with the same parent, they must 
be identical. These trees are shown to be identical by induction.  Given that 
the top nodes are the same coloring, with a common parent, both must have the 
same number and color children.  There exists an ordering of the children such 
that the children of $V_b$ match exactly the children of $V_c$ because they are 
trees, so they are planar and acyclic.  There must also exist an ordering of 
the children of the children of $V_b$ and $V_c$ by the same principles.  This 
process continues until the trees are the exact same in every level.  Since the 
trees are the same but there is some vertex or edge in one but not the other,
to satisfy $V_a$, there is a contraction.$\qed$\\


\begin{mincolor}
If some vertex, $V_i$ connects to two or more vertices of the same coloring in a
planar-expansion then the expansion is not minimal.
\end{mincolor}
\paragraph{Proof: }
Given some graph, \emph{G}, where some vertex, $V_i$, connects 
to two vertices of the same coloring, it can be made smaller while maintaining 
its planar-expansion property with the following algorithm:...\\

\begin{duplication}
Given some non-planar graph, G, if some vertex, $V_i$, must be duplicated to 
create a minimum planar-expansion, then all vertices must be duplicated.
\end{duplication}
\paragraph{Proof: }
For the sake of contradiction, assume not and that 
vertex $V_a$ is not duplicated to create a planar-expansion. $V_i$, however, was
duplicated. This must happen because \emph{G} is non-planar and so at least one 
vertex must be duplicated. Let us say that set \emph{A} is the set of vertices 
that $V_i$ connects to in \emph{G}. It follows then that there are two sets of 
vertices, \emph{B} and \emph{C}, that are independent of each other such that 
\emph{B} $\cup$ \emph{C}=\emph{A} such that \emph{B} can connect to $V_i$,
 planarly, but not to $V^{'}_{i}$. Conversely, \emph{C} can connect to 
 $V^{'}_{i}$ planarly. At this point, \emph{B} and \emph{C} must be duplicated
 to planarly satisfy $V^{'}_i$ and $V_i$, respectively. $V_i$ is then 
 connected to \emph{B} and \emph{$C^{'}$}, and vise versa. The ordering of
 \emph{B} $\cup$ \emph{$C^{'}$} and \emph{C} $\cup$ \emph{$B^{'}$} must be 
 unique.  If they were the same then the tree would be unnecessary: $V_i\ and\ 
 V^{'}_{i}$ are required by the same vertex (not multiple as $V_i$ is the first
 instance of a duplication) so with their parent as the root, the subgraph
 including $V_i\ and\ V^{'}_{i}$ would violate the Unnecessary Trees lemma.

\end{document}
